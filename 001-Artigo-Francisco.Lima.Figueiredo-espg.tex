%% abtex2-modelo-artigo.tex, v<VERSION> laurocesar
%% Copyright 2012-<COPYRIGHT_YEAR> by abnTeX2 group at http://www.abntex.net.br/ 
%%
%% This work may be distributed and/or modified under the
%% conditions of the LaTeX Project Public License, either version 1.3
%% of this license or (at your option) any later version.
%% The latest version of this license is in
%%   http://www.latex-project.org/lppl.txt
%% and version 1.3 or later is part of all distributions of LaTeX
%% version 2005/12/01 or later.
%%
%% This work has the LPPL maintenance status `maintained'.
%% 
%% The Current Maintainer of this work is the abnTeX2 team, led
%% by Lauro César Araujo. Further information are available on 
%% http://www.abntex.net.br/
%%
%% This work consists of the files abntex2-modelo-artigo.tex and
%% abntex2-modelo-references.bib
%%

% ------------------------------------------------------------------------
% ------------------------------------------------------------------------
% abnTeX2: Modelo de Artigo Acadêmico em conformidade com
% ABNT NBR 6022:2018: Informação e documentação - Artigo em publicação 
% periódica científica - Apresentação
% ------------------------------------------------------------------------
% ------------------------------------------------------------------------

\documentclass[
	% -- opções da classe memoir --
	article,			% indica que é um artigo acadêmico
	11pt,				% tamanho da fonte
	oneside,			% para impressão apenas no recto. Oposto a twoside
	a4paper,			% tamanho do papel. 
	% -- opções da classe abntex2 --
	%chapter=TITLE,		% títulos de capítulos convertidos em letras maiúsculas
	%section=TITLE,		% títulos de seções convertidos em letras maiúsculas
	%subsection=TITLE,	% títulos de subseções convertidos em letras maiúsculas
	%subsubsection=TITLE % títulos de subsubseções convertidos em letras maiúsculas
	% -- opções do pacote babel --
	english,			% idioma adicional para hifenização
	brazil,				% o último idioma é o principal do documento
	sumario=tradicional
]{abntex2}


% ---
% PACOTES
% ---

% ---
% Pacotes fundamentais 
% ---
\usepackage{lmodern}			% Usa a fonte Latin Modern
\usepackage[T1]{fontenc}		% Selecao de codigos de fonte.
\usepackage[utf8]{inputenc}		% Codificacao do documento (conversão automática dos acentos)
\usepackage{indentfirst}		% Indenta o primeiro parágrafo de cada seção.
\usepackage{nomencl} 			% Lista de simbolos
\usepackage{color}				% Controle das cores
\usepackage{graphicx}			% Inclusão de gráficos
\usepackage{microtype} 			% para melhorias de justificação

\usepackage{amsmath}							% Pacote matemático
\usepackage{amssymb}							% Pacote matemático
\usepackage{amsfonts}							% Pacote matemático

% ---

% ---
% Pacotes adicionais, usados apenas no âmbito do Modelo Canônico do abnteX2
% ---
\usepackage{lipsum}				% para geração de dummy text
\usepackage[table,xcdraw,svgnames]{xcolor}

\usepackage{float}
\usepackage{placeins}

% Personalizações de cores da UDESC
\definecolor{CapaAmareloUDESC}{RGB}{243,186,83}		% Especializacao
\definecolor{CapaVerdeUDESC}{RGB}{0,112,52}			% Mestrado
\definecolor{CapaVermelhoUDESC}{RGB}{171,35,21}		% Doutorado
\definecolor{CapaAzulUDESC}{RGB}{38,54,118} 		% Pós-Doutorado


% ---

% ---
% Pacotes de citações
% ---
\usepackage[brazilian,hyperpageref]{backref}	 % Paginas com as citações na bibl
\usepackage[alf]{abntex2cite}	% Citações padrão ABNT
% ---

% ---
% Pacotes de Listagens de código
% ---
\usepackage{listings}
% ---

\label{__:programinhas}
% ---
% Configurações do pacote backref
% Usado sem a opção hyperpageref de backref
\renewcommand{\backrefpagesname}{Citado na(s) página(s):~}

% Texto padrão antes do número das páginas
\renewcommand{\backref}{}

\newcommand{\MONTH}{%
	\ifcase\the\month
	\or JAN% 1
	\or FEV% 2
	\or MAR% 3
	\or ABR% 4
	\or MAI% 5
	\or JUN% 6
	\or JUL% 7
	\or AUG% 8
	\or SET% 9
	\or OUT% 10
	\or NOV% 11
	\or DEZ% 12
	\fi}
\makeatletter

% Define os textos da citação
\renewcommand*{\backrefalt}[4]{
	\ifcase #1 %
		Nenhuma citação no texto.%
	\or
		Citado na página #2.%
	\else
		Citado #1 vezes nas páginas #2.%
	\fi}%


% Citação online --- MODIFICAR ---
\newcommand{\citeaa}[1]{\citeauthoronline{#1}~(\citeyear{#1})}

\newenvironment{Figure}
	{\par\medskip\noindent\minipage{\linewidth}}
	{\endminipage\par\medskip}


% Força autores do artigo a ficar para o lado direito da folha
\usepackage{titling}
\preauthor{\begin{flushright}
		\large \lineskip 0.5em%
}
\postauthor{\end{flushright}}


%-----------------------------------------
% (1) simple command for print or not
%-----------------------------------------
\usepackage{ifthen}
\newcommand{\ImprimirSimOuNao}[2][Sim]
{
	\ifthenelse{\equal{#1}{Sim}}{#2}{}
}


\label{__:Título}
% --- Informações de dados para CAPA e FOLHA DE ROSTO ---
\titulo{
	Usando regressão linear para estimar os pesos de questões de matemática do ENEM na nota final do aluno
}

\tituloestrangeiro{
%	Analysis of student performance in solving ENEM issues, from the perspective of Polya, in the light of statistical parameters obtained from the ENEM microdata from 2015 to 2019.
}

%\usepackage{titling}
\makeatletter
\let\@fnsymbol\@arabic
\setlength{\skip\footins}{2pc}
\setlength{\footnotesep}{1.1pc}
\makeatother


\label{__:Dados-do-Documento}
\autor{
	Francisco Lima Figueiredo\thanks{Aluno concluinte do curso de pós-graduação em MATEMÁTICA E ESTATÍSTICA, Universidade ESPG. \href{http://lattes.cnpq.br/2974257608350963}{Currículo Lattes <http://lattes.cnpq.br/2974257608350963>}. Matrícula 102020777. }{} 
}

\local{Brasil}
\data{\the\day/\MONTH/\the\year}
% ---

% ---
% Configurações de aparência do PDF final

% alterando o aspecto da cor azul
\definecolor{blue}{RGB}{41,5,195}

% informações do PDF
\makeatletter
\hypersetup{
	%pagebackref=true,
	pdftitle={\@title}, 
	pdfauthor={\@author},
	pdfsubject={Modelo de artigo científico com abnTeX2},
	pdfcreator={LaTeX with abnTeX2},
	pdfkeywords={abnt}{latex}{abntex}{abntex2}{atigo científico}, 
	colorlinks=true,       		% false: boxed links; true: colored links
	linkcolor=NavyBlue,          	% color of internal links
	citecolor=NavyBlue,        		% color of links to bibliography
	filecolor=magenta,      		% color of file links
	urlcolor=NavyBlue,
	bookmarksdepth=4
}
\makeatother
% --- 

% ---
% compila o indice
% ---
\makeindex
% ---

% ---
% Altera as margens padrões
% ---
\setlrmarginsandblock{3cm}{3cm}{*}
\setulmarginsandblock{3cm}{3cm}{*}
\checkandfixthelayout
% ---

% --- 
% Espaçamentos entre linhas e parágrafos 
% --- 

% O tamanho do parágrafo é dado por:
\setlength{\parindent}{1.3cm}

% Controle do espaçamento entre um parágrafo e outro:
\setlength{\parskip}{0.2cm}  % tente também \onelineskip

% Espaçamento simples
\SingleSpacing


% -----------------------------------------------------------------
% pacote interno
\usepackage{modulos/trabalhos.academicos}


% ----
% Início do documento
% ----
\begin{document}
	
	% Seleciona o idioma do documento (conforme pacotes do babel)
	%\selectlanguage{english}
	\selectlanguage{brazil}
	
	% Retira espaço extra obsoleto entre as frases.
	\frenchspacing 
	
	% ----------------------------------------------------------
	% ELEMENTOS PRÉ-TEXTUAIS
	% ----------------------------------------------------------
	
	%---
	%
	% Se desejar escrever o artigo em duas colunas, descomente a linha abaixo
	% e a linha com o texto ``FIM DE ARTIGO EM DUAS COLUNAS''.
	% \twocolumn[    		% INICIO DE ARTIGO EM DUAS COLUNAS
	%
	%---

	\ImprimirSimOuNao[Não]{\imprimircapaArtigo}			% Capa UDESC para TCC

	
	% página de titulo principal (obrigatório)
	\maketitle	

	% titulo em outro idioma (opcional)

	
	% resumo em português
	\begin{resumoumacoluna}
		
		O presente artigo apresenta os resultados preliminares de um estudo para tentar responder a seguinte pergunta: É possível atribuir um peso, uma nota a cada questão de matemática do ENEM de forma a apresentar os resultados ao aluno em termos equivalentes à de uma avaliação clássica.
		
		Foram feitas análises estatísticas a partir dos microdados do ENEM ...
		
		
		\vspace{\onelineskip}
		
		\noindent
		\textbf{Palavras-chave}: ENEM. Matemática. TRI - Teoria de Resposta ao Item. Mínimos Quadrados Ordinários. Avaliação Educacional.
	\end{resumoumacoluna}
	
	
	\ImprimirSimOuNao[Não]{
			% resumo em inglês
		\renewcommand{\resumoname}{Abstract}
		\begin{resumoumacoluna}
			\begin{otherlanguage*}{english}
				According to ABNT NBR 6022:2018, an abstract in foreign language is optional.
				
				\vspace{\onelineskip}
				
				\noindent
				\textbf{Keywords}: latex. abntex.
			\end{otherlanguage*}  
		\end{resumoumacoluna}
		
		% ]  				% FIM DE ARTIGO EM DUAS COLUNAS
	}


	% ---
	\ImprimirSimOuNao[Não]{
		\begin{center}\smaller
			\textbf{Data de submissão e aprovação}: elemento obrigatório. Indicar dia, mês e ano
			
			\textbf{Identificação e disponibilidade}: elemento opcional. Pode ser indicado 
			o endereço eletrônico, DOI, suportes e outras informações relativas ao acesso.
		\end{center}
	}
	
	% ----------------------------------------------------------
	% ELEMENTOS TEXTUAIS
	% ----------------------------------------------------------
	\textual
	
	% ----------------------------------------------------------
	% Introdução
	% ----------------------------------------------------------
	\section{Introdução}
	
		Obviamente o ENEM se tornou hoje fonte de preocupação de alunos e pais em todo país. As notas do ENEM decidem futuros e rumos familiares, uma vez que suas notas determinam o ingresso na rede de ensino superior, pública e particular, bem como definem valores e limites de financiamento para estudantes e avaliação do ensino.
		
		Por mais inegável que existe uma certa aceitação de como o ENEM é feito e sua avaliação, não podemos negar que o seu esquema de notas à olhos leigos assusta, confunde e deixa sem ferramentas os professores, além daquelas obtidas pelos organizadores e pelo empirismo, para argumentar acerca de quanto vale cada questão.
		
		O aluno é, desde criança, imerso em um sistema de notas 'linear' onde cada questão tem um peso e/ou que a quantidade de questões acertadas está diretamente ligada ao seu desempenho.

		Ao fazer o ENEM, os alunos ficam as "cegas" antes, durante e após o certame. E uma simples análise de vídeos no youtube como visto em \citeaa{PauloValim2019}, \citeaa{TVBrasil2021} e \citeaa{MEC2020} as explicações, embora seja didática e aparentemente simples, não respondem as especificidades da Teoria da Resposta ao Item. Cabe ressaltar que os vídeos citados tem imprecisões quando comparadas ao cerne da TRI, seja por medo ou desconhecimento matemático-estatístico necessário para compreender, como brilhantemente explicado em \citeaa{Pasquali2018} em relação aos colegas do ramo das ciência sociais: 
		
		\begin{citacao}
				Além disso, diante do receio da maioria dos cientistas sociais, em particular do psicólogo e do pedagogo, frente ao pensamento 	matemático, bem como o fraco preparo desses profissionais nas áreas da matemática, este livro procura enveredar o mínimo na sofisticação matemática e estatística que a TRI pode assumir. 		
				
				Obviamente, não é possível escapar totalmente do pensar matemático nessa área da medida, pois seria utilizar o número, o objeto específico das matemáticas, sem fazer uso dos princípios e métodos dessas. Quem entende e trabalha o número são, necessariamente, as matemáticas. Então não há como eliminá-las no tratamento dos dados empíricos expressos via medida. Contudo para fazer uso inteligente dos princípios e da tecnologia da TRI não é necessário entrar nas altas sofisticações matemáticas e estatísticas que eles permitem. Evidentemente, quem é capaz de seguir por esse caminho tem maiores vantagens na intelecção da problemática psicométrica e do seu ramo mais sofisticado, que é a TRI.
		\end{citacao}
	
		O presente trabalho foca em encontrar pesos para cada questão do ENEM de um determinado ano de forma que o aluno possa, mesmo que por aproximação, verificar o peso de cada questão na sua nota final.
		
	
	\section{Referencial Teórico}
	
		A teoria envolvida nesse artigo é bem ampla e está melhor detalhada na trabalho, em formato de tese, auxiliar a esse artigo em \citeaa{Figueiredo2021Tese}.
	
		\subsection{Avaliação - Modelos matemáticos}
			Segundo \citeaa{AlvesFerreira2018}, A avaliação sempre foi um dos pontos críticos do sistema educacional, pois os métodos utilizados largamente não permitem medir o quão verdadeiro é o conhecimento do indivíduo, de modo que, os testes estão sofrendo revisões e novos instrumentos são criados para resolver e ajudar em uma adequada avaliação, principalmente na área educacional. Os resultados dados pelo número ou pelo percentual de acertos em uma escala própria em geral não diz muito sobre o respondente e prejudica o julgamento sobre um determinado conteúdo. Segundo \citeaa{Klein2013} costuma-se interpretar que as notas acima de setenta por cento são consideradas adequadas e as abaixo de cinquenta por cento como fracasso. Evidente que, quando as notas são abaixo do esperado, associa-se que, provavelmente a prova estava num nível acima dos alunos ou que os itens estavam em um grau de dificuldade demasiado. Esta percepção não leva em consideração o estado de espirito, a habilidade, a competência do aluno ou a discriminação dos itens.
			
			Para lidar com esse problema, modelos matemáticos foram criados, entre eles a Teoria Clássica dos Testes - TCT e a Teoria de Resposta ao Item - TRI. A teoria TCT expressa escores brutos ou padronizados e tem sido utilizada nos processos de avaliação e seleção de indivíduos e, as análises estão sempre ligadas à prova como um todo, ficando inviável a comparação entre itens e indivíduos. Já a teoria TRI propõe modelos que analisa características do indivíduo que não podem ser medida diretamente. O método sugere formas de apresentar a probabilidade de um indivíduo dar uma certa resposta, levando em consideração suas habilidades na área de conhecimento. A grande vantagem da TRI sobre a TCT é que essa teoria permite a comparação entre populações, desde que as provas submetidas tenham itens comuns e ainda, a comparação entre os resultados dos indivíduos submetidos à provas diferentes em épocas diferentes. Essa vantagem se dá porque esta teoria tem como elemento central o item e não a prova como um todo. Assim, a introdução da TRI trouxe muitas vantagens sobre o método tradicional, pois permite estimar e comparar os resultados dos alunos, mesmo que respondam itens diferentes, colocando-os em uma mesma escala (\citeaa{AlvesFerreira2018}).
			
			\subsubsection{Teoria Clássica dos Testes - TCT}
			
				Segundo \citeaa{Klein2013} a Teoria Clássica dos Testes - TCT, orientou por
				muito tempo o desenvolvimento dos testes educacionais, psicológicos, avaliativos e etc. A
				TCT baseia-se no escore bruto de cada indivíduo, ou seja, o resultado é obtido apenas
				comparando a quantidade de itens respondidos corretamente por cada indivíduo, logo a
				habilidade é estimada pelo número de acertos, sendo X escore observável obtido por um
				estudante em um teste.
				
				A confiabilidade depende dos alunos testados, do tamanho e dos itens do teste.
				Itens com coeficiente ponto bisserial, correlação entre acerto e número de acertos na prova,
				baixo ou negativo acrescentam pouco à confiabilidade.
				
				
				
				A Teoria Clássica dos Testes, apesar de amplamente utilizada nos diversos tipos de avaliação, têm limitações, dentre as quais foram listadas por \citeaa{Klein2013}
				
				\begin{itemize}
					\item As estatísticas que descrevem os itens de teste dependem do grupo de estudantes
					que fazem o teste;
					
					\item Os escores de teste que descrevem o desempenho dos alunos dependem dos itens
					apresentados aos alunos;
					
					\item A Teoria Clássica dos Testes só pode ser utilizada em situações nas quais todos os
					alunos fazem o mesmo teste (ou formas “paralelas” de teste);
					
					\item A Teoria Clássica dos Testes não fornece um modelo do desempenho de um aluno
					em um item.
					
					\item A maioria das aplicações da Teoria Clássica dos Testes assume incorretamente que
					os erros de medida têm a mesma variabilidade para todos os alunos.
					
					
				\end{itemize}

			\subsubsection{Teoria de Resposta ao Item - TRI}

				Segundo \citeaa{Klein2013} a Teoria Clássica dos Testes - TCT, orientou por
				muito tempo o desenvolvimento dos testes educacionais, psicológicos, avaliativos e etc. A
				TCT baseia-se no escore bruto de cada indivíduo, ou seja, o resultado é obtido apenas
				comparando a quantidade de itens respondidos corretamente por cada indivíduo, logo a
				habilidade é estimada pelo número de acertos, sendo X escore observável obtido por um
				estudante em um teste.
				
				A confiabilidade depende dos alunos testados, do tamanho e dos itens do teste.
				Itens com coeficiente ponto bisserial, correlação entre acerto e número de acertos na prova,
				baixo ou negativo acrescentam pouco à confiabilidade.	
				
				O TRI trouxe diversos avanços, como mencionam \citeaa{Pasquali2018} \nocite{PASQUALI2003} referenciando-se à \citeaa{Hambleton1991}:
				
				\begin{enumerate}[label=\alph*)]
					\item O \textit{cálculo do nível de aptidão do sujeito} é independente da lista de itens utilizados: demonstra-se
					que a habilidade do sujeito é independente do teste (not test-dependent). Na
					Psicometria Clássica, o escore do sujeito dependia e variava segundo a facilidade do teste aplicado
					, ou dos erros produzidos. Assim, tais escores tinham base comparativa e, mesmo após ajustes, os escores ainda
					não permitiam comparações, sobretudo porque os testes produziam diferenças nas
					variâncias nos erros de medida. No caso da TRI, não importa as questões ou conjunto de
					questões que você utilize, obviamente que estejam medindo o mesmo traço latente, eles
					irão produzir o mesmo nível de aptidão do sujeito, dentro dos sempre
					presentes erros de medida em qualquer ramo da ciência;
					
					\item \textit{O cálculo ou estimativa dos parâmetros da função de probabilidade dos itens} (dificuldade e discriminação) é realizada independe da amostra de sujeitos utilizada: diz-se que são independentes dos
					sujeitos ou não dependente de grupo (not group-dependent). Na teoria clássica, os parâmetros dependiam demais dos indivíduos amostrados terem maior ou menor aptidão;
					
					\item A TRI permite \textit{comparabilidade dos itens com a aptidão do sujeito}. em outras palavras que se avalia
					a proficiência de um sujeito utilizando-se questões com dificuldade tal que se situam ao redor do requerido
					tamanho da aptidão do sujeito, sendo, assim, possível utilizar desde questões mais fáceis para sujeitos com habilidades inferiores e questões com dificuldade mais elevada para sujeitos com maior proficiência, e ainda sim
					produzindo scores comparáveis em ambos os casos. Na psicometria clássica, era aplicado o mesmo teste sempre, hermeticamente fechado, para todos os sujeitos, de tal sorte que, se o teste fosse fácil, seriam bem avaliado os sujeitos de aptidão menor e mal avaliados os sujeitos de aptidão superior e, caso o teste fosse difícil, faria o contrário.
					
					\item A TRI é um modelo que não precisa fazer suposições que aparentam serem
					improváveis, tais como os erros de medida serem iguais para todos os testandos;
					
					\item A TRI não necessita trabalhar com testes estritamente paralelos como exige a
					psicometria clássica.
				\end{enumerate}
			
				O TRI por sí só merece uma tese a parte, mas basicamente a teoria se baseia em algumas premissas:
				
				\begin{enumerate}[label=\alph*)]
					\item A intenção do modelo é mensurar a habilidade, traço latente ou aptidão que ele possua, seja numa matéria escolar ou avaliação psíquica. A essa habilidade vamos atribuir um score $ \theta_{j} $\footnote{Leia-se habilidade do indivíduo \textit{i}}, mensurável numa escala de média 0 e desvio padrão 1. No ENEM, por exemplo, eles ajustam para, a grosso modo, uma nota com média 500 e desvio padrão 100 usando algo parecido com $ N_{j} = 100 \theta_{j} + 500 $ \footnote{Leia-se: A nota do indivíduo \textit{j} é 100x a habilidade $ \theta $ do indivíduo somado a 500 que é a média esperada}
					
					\item Ao efetuar uma prova, para cada questão é mensurado ou atribuído uma função de probabilidade para que, com o indivíduo de habilidade $ \theta_{j} $ acerte a questão $ U_{i} $ que é definido, para o modelo de 3 parâmetros, como:
					
					
					\[ P(U_{ji} = 1 | \theta_{j}) = c_{i} + (1 - c_{i}) \cdot \dfrac{1}{1 + e^{-D a_{i} ( \theta_{i} - b_{i})}} \]
					
					Onde $ P(U_{ji} = 1 | \theta_{j}) $ é a probabilidade do indivíduo \textit{j} acertar a questão \textit{i}, dado sua habilidade ou traço latente $ \theta_{j} $, $ c_{i} $ é o \textbf{parâmetro de acerto ao acaso} do item \textit{i} que representa a probabilidade de indivíduos com baixa ou nenhuma habilidade $ \theta_{j} $ responderem corretamente o item i (muitas vezes referido como a probabilidade de acerto casual) e $ b_{i} $ é o \textbf{parâmetro de dificuldade} (ou de posição) do item i, medido na mesma escala da habilidade. Em outras palavras, esse parâmetro mede a habilidade $ \theta_{j} $ necessária ao indivíduo para que tenha 50\% de chance de acertar a questão; e $ a_{i} $ é o \textbf{parâmetro de discriminação} (ou de inclinação) do item i, com valor proporcional à inclinação da Curva Característica do Item — CCI no ponto $ b_{i} $. Segundo \citeaa{SilvaGomes2014} para entendermos melhor esse parâmetro, devemos ter clareza a respeito do que venha ser a discriminação de um item. É plausível pensarmos que numa prova feita com vários respondentes com habilidades distintas, alguns itens serão considerados fáceis pelos indivíduos com uma proficiência alta, mas podem ser considerados difíceis pelos indivíduos com uma baixa proficiência. A discriminação é exatamente essa característica do item capaz de diferenciar indivíduos com habilidades distintas.

					\begin{figure}[htb]
						\centering
						\includegraphics[width=0.5\linewidth]{figuras/fig001-curva.caracteristica}
						\caption[CCI - Curva Característica do Item]{CCI - Curva Característica do Item, disponível para manipulação em <\href{https://www.geogebra.org/classic/nq8dk5x9}{https://www.geogebra.org/classic/nq8dk5x9}>. \\
							Fonte: Autor}
						\label{fig:fig01-tri}
					\end{figure}
					
					Essa curva de probabilidade é caracterizada por uma curva em formato de S, basicamente a medida que aumenta a habilidade $\theta$ do aluno, maior a probabilidade de acerto da questão $Q_i$, como demonstra a figura~\ref{fig:fig01-tri}  
					
				
				\end{enumerate}
			
										
		\subsection{Mínimos Quadrados Ordinários}
			De acordo com \citeaa{WikipediaMMQ2021}, o Método dos Mínimos Quadrados (MMQ), ou Mínimos Quadrados Ordinários (MQO) ou OLS (do inglês Ordinary Least Squares) é uma técnica de otimização matemática que procura encontrar o melhor ajuste para um conjunto de dados tentando minimizar a soma dos quadrados das diferenças entre o valor estimado e os dados observados (tais diferenças são chamadas resíduos).
			
			Para o caso geral de 2 dimensões ($ \mathbb{R}^2 $), temos um conjunto finito de n dados $ \mathbb{D} = \left\lbrace (x_i,y_i) | 1 \leqslant i \leqslant n \right\rbrace $, e uma função parametrizável $ f(x, \alpha, \beta, \gamma, ...) $, onde $ \alpha, \beta, \gamma, ... $ são parâmetros que se desejam achar com o objetivo de minimizar o somatório dos quadrados dos resíduos $r_i = y_i - f(x_ i, \alpha, \beta, \gamma, ...) $ ou seja :
			
			\[ \text{minimizar  } S_r = \sum_{i=1}^n \left[ y_ i - f(x_ i, \alpha, \beta, \gamma, ...) \right]^2  \]
			
			Essa minimização pode ser obtida, derivando-se em relação aos parâmetros e igualando a zero, ou seja
			
			\[ \dfrac{\partial S_r}{\partial \alpha} = 0 , \dfrac{\partial S_r}{\partial \beta} = 0 , \dfrac{\partial S_r}{\partial \gamma} = 0 , ... \]
			
			Na vida prática, usamos algoritmos numéricos, muito populares em pacotes de estatística como Python e R, para o cálculo aproximando desses parâmetros, como usado nesse artigo. 
			
	\section{Metodologia}
		
		Foi utilizado o Python como ferramenta de análise de dados e técnicas de data science para avaliar os microdados do ENEM em busca de insights e testes de hipotese sobre os dados em busca de informações úteis sobre o assunto.

		\subsection{Catalogação e classificação das questões do ENEM}
			Para iniciar os estudos e coleta e gravação das informações, Anteriormente ao processamento acima descrito, foi criado um banco de dados Access \footnote{Microsoft Access, integrante do pacote Office} contendo todas as questões dos ENEM, contendo, para cada questão:
			\begin{enumerate}
				\item \textbf{Imagem da questão}: cada questão possui uma imagem fidedigna, as vezes ajustada, da questão no caderno de questões azul de cada prova.
				
				\item \textbf{Habilidade}: habilidade avaliada na questão, conforme parâmetros retirado dos microdados e comparado com o documento \textbf{Matriz de referência ENEM}, conforme \citeaa{INEPMatriz2020}.
				
				\item \textbf{Gabarito da questão}: tirada dos microdados do ENEM, equivale a alternativa considerada correta para fins de constatação estatística
				
				\item \textbf{Conhecimentos mapeados para resolução} - lista padronizada de conhecimentos, mapeados a partir do conhecimento necessários para resolver de forma simples a questão, baseados no conhecimento do autor.
				
				\item \textbf{Mapa de Respostas} - lista com o número da questão, conforme a prova. Já reservada futuramente caso a questão mude a ordem das alternativas.
				
				\item \textbf{Dificuldade} - Classificação de dificuldade que o aluno teria para resolução da questão, baseado na opinião do autor baseada em sua experiência e na classificação de ao menos 2 professores que corrigiram as questões em vídeos do Youtube.

			\end{enumerate}
		
			Além das variáveis acima mapeados, foram criadas campos nesse arquivo de banco de dados para recepcionar as variáveis geradas nesse e em outros estudos, sendo que para esse artigo nosso escopo será o campo \textbf{Peso Nota} ou registrado em \citeaa{Figueiredo2021Tese} denominado \textbf{Peso do Item na Nota Final de Matemática}.
		
			\begin{figure}[H]
				\centering
				\includegraphics[width=0.95\linewidth]{figuras/0001-banco-de-questoes.png}
				\caption[Banco de Questões]{Banco de Questões desenvolvido pelo Autor para catalogação das variáveis de cada questão, disponível no link \url{https://github.com/ChicoFigueiredo/Estacio-TCC-Estacio-Matematica/tree/main/03-pesquisa/banco.questoes} para download}
				\label{fig:0001-banco-de-questoes}
			\end{figure}
		
			Todos os resultados desse banco estão disponíveis de maneira organizada em \citeaa{Figueiredo2021Tese} no Anexo C, disponível em \url{https://github.com/ChicoFigueiredo/Estacio-TCC-Estacio-Matematica/tree/main/03-pesquisa/banco.questoes} conforme exemplo na figura ~\ref{fig:figura-defesa-exemplo}:
			
			\begin{figure}[H]
				\centering
				\includegraphics[width=0.8\linewidth]{figuras/figura-defesa-exemplo}
				\caption[Exemplo de Questão]{Questão detalhada em \citeaa{Figueiredo2021Tese} elaborada pelo autor, página 202}
				\label{fig:figura-defesa-exemplo}
			\end{figure}
						
			Para geração rápida de informações, também foi produzida a partir do banco de dados também uma planilha Excel, com informações mais importantes, disponível em \url{https://github.com/ChicoFigueiredo/Estacio-TCC-Estacio-Matematica/blob/main/03-pesquisa/banco.questoes/Mapa_Questoes.xlsm?raw=true}
			
	
		\subsection{Processamento e análise dos microdados ENEM}
			Como cita \citeaa{Morita2021}, a etapa mais demorada e trabalhosa do projeto é a obtenção e tratamento de dados. Normalmente, quando estudamos Data Science, as bases de dados usadas estão prontas para análise e são de fácil acesso. Na prática, é o oposto! Dados têm diferentes fontes e formatos. Podemos analisar tabelas, imagens, áudios, textos vindos de redes sociais, sites, bancos de dados, pesquisas, documentos digitalizados, etc. Além disso, eles estão longe de prontos para serem analisados: precisam antes serem tratados e integrados.
			
			Os microdados do Enem são o menor nível de desagregação de dados recolhidos por meio do exame. Eles atendem a demanda por informações específicas ao disponibilizar as provas, os gabaritos, as informações sobre os itens, as notas e o questionário respondido pelos inscritos no Enem (\citeaa{InepENEM2021}).
			
			Foram realizados o download dos arquivos de microdados dos ENEM entre os anos de 2016 a 2021 que consistem, basicamente, de arquivos CSV \footnote{CSV = Comma-separated values, valores separados por vírgula, formato de arquivo de texto de formato regulamentado pelo \href{https://datatracker.ietf.org/doc/html/rfc4180}{RFC 4180}, que faz uma ordenação de bytes ou um formato de terminador de linha, separando valores com vírgulas (podendo ser separado por ponto-e-virgula). Ele comumente é usado em softwares de escritório, tais como o Microsoft Excel e o LibreOffice Calc, ou de troca de dados estatístico (R ou Python).} gigantescos, que variam na casa de 3 Gb \footnote{3 Gb = $3 \cdot 1024 \cdot 1024 \cdot 1024 \approx 3,75  $ bilhões de caracteres} de tamanho cada arquivo.
			
			Para fins desse estudo, foi feito um tratamento dos dados para normalização e padronização dos dados, pois algumas variáveis se alteraram de um ano para outro, tais como posição das questões conforme a cor do caderno, a presença ou não, ou posição, de algumas variáveis sociais, entre outros tratamentos. Isso está detalhado nos notebooks python (extensão ipynb) presente no endereço \url{https://github.com/ChicoFigueiredo/Estacio-TCC-Estacio-Matematica/tree/main/03-pesquisa} ou detalhado em \citeaa{Figueiredo2021Tese}.

			Em resumo, primeiro foi montado um programa em um notebook python para executar o tratamento de dados e normalização de algumas tabelas como demonstra o ambiente presente na figura~\ref{fig:0003-programa-python-01}

			\begin{figure}[H]
				\centering
				\includegraphics[width=0.95\linewidth]{figuras/0003-Programa-python-01}
				\caption[Ambiente de desenvolvimento Jupyter Lab, contendo o programa encontrado em]{Ambiente de desenvolvimento Jupyter Lab, contendo o programa de limpeza de dados encontrado em \url{https://github.com/ChicoFigueiredo/Estacio-TCC-Estacio-Matematica/blob/main/03-pesquisa/01-ENEM.Microdados-Tratamento.ipynb}
					Fonte: O Autor.}
				\label{fig:0003-programa-python-01}
			\end{figure}
			
			Em seguida, foi montado um segundo programa para gerar as informações necessárias para determinar os pesos de cada questão, em cada ano, conforme a figura~\ref{fig:0003-programa-python-02}
			
			\begin{figure}[H]
				\centering
				\includegraphics[width=0.95\linewidth]{figuras/0003-Programa-python-02}
				\caption[Ambiente de desenvolvimento Jupyter Lab, contendo o programa encontrado em]{Ambiente de desenvolvimento Jupyter Lab, contendo o programa de regressão linear encontrado em \url{https://github.com/ChicoFigueiredo/Estacio-TCC-Estacio-Matematica/blob/main/03-pesquisa/02-ENEM.Geração-Estatísticas.ipynb}. Fonte: O Autor.}
				\label{fig:0003-programa-python-02}
			\end{figure}
			
			O mesmo programa gravou os dados no banco de dados access relatado acima.
			

		\subsection{Análise estatística e obtenção de parâmetros das questões}
			
			Para a modelagem do problema, foi suposto que um aluno $i$ consegui a nota ponderada $N_i$ com o gabarito de questões marcadas $( x_{i,1}, x_{i,2}, x_{i,3}, ..., x_{i,45} )$ onde $x_{i,j}$ são variáveis dummies que possuem o valor $x_{i,j} = 1$ caso o aluno acerte ou a questão seja anulada e $x_{i,j} = 0$ caso ele erre ou deixe em branco.
			
			O objetivo é encontrar os pesos $\rho_1,\rho_2,\rho_3,...,\rho_{45}$ que permitam assertivamente ter a nota $N_i$ a partir do gabarito do aluno, com o menor erro $\varepsilon_i$ possível para o aluno.
			
			
			\[  N_i = \rho_1 \cdot x_{i,1} + \rho_2 \cdot x_{i,2} + \rho_3 \cdot x_{i,3} + ... + \rho_{45} \cdot x_{i,45} + \varepsilon_i \]

			 ou resumidamente

			\[ N_i = \sum_{j=1}^{45} \rho_j \cdot x_{i,j}  + \varepsilon_i   \]
							
			Em termos práticos, a otimização é:

			\[ \text{minimizar  } \varepsilon_i = N_i - \sum_{j=1}^{45} \rho_j \cdot x_{i,j}   \]
				
			Usando o programa montado no arquivo \url{https://github.com/ChicoFigueiredo/Estacio-TCC-Estacio-Matematica/blob/main/03-pesquisa/02-ENEM.Geração-Estatísticas.ipynb} foi possível, a partir dos dados processados, conseguir facilmente os pesos $\rho_j$ de cada questão usando o método de mínimos quadrados ordinários.
			
			Para demonstrar o poder de aproximação desse modelo, o output abaixo é o sumário da regressão gerado pelo framework statsmodels.
			
			\begin{verbatim}
				                            OLS Regression Results                            
				==============================================================================
				Dep. Variable:             NU_NOTA_MT   R-squared:                       0.971
				Model:                            OLS   Adj. R-squared:                  0.971
				Method:                 Least Squares   F-statistic:                 1.493e+06
				Date:                Sat, 08 Jul 2023   Prob (F-statistic):               0.00
				Time:                        21:32:08   Log-Likelihood:            -8.4452e+06
				No. Observations:             1938906   AIC:                         1.689e+07
				Df Residuals:                 1938861   BIC:                         1.689e+07
				Df Model:                          44                                         
				Covariance Type:            nonrobust                                         
				==============================================================================
				coef    std err          t      P>|t|      [0.025      0.975]
				------------------------------------------------------------------------------
				const        332.7341      0.035   9427.572      0.000     332.665     332.803
				Q_111459       2.8457      0.038     74.448      0.000       2.771       2.921
				Q_111476      16.1274      0.030    535.996      0.000      16.068      16.186
				Q_111521      27.5088      0.028    976.559      0.000      27.454      27.564
				Q_111716       1.2208      0.036     34.369      0.000       1.151       1.290
				Q_117624      12.3872      0.036    345.981      0.000      12.317      12.457
				Q_117674   -7.929e-15   3.38e-17   -234.558      0.000   -7.99e-15   -7.86e-15
				Q_117675       6.5048      0.032    203.341      0.000       6.442       6.567
				Q_117692      15.9844      0.033    484.028      0.000      15.920      16.049
				Q_117731       0.3077      0.035      8.738      0.000       0.239       0.377
				Q_117831       3.4616      0.033    104.379      0.000       3.397       3.527
				Q_117832       6.1131      0.036    170.013      0.000       6.043       6.184
				Q_117852      12.8248      0.030    422.071      0.000      12.765      12.884
				Q_117902      21.6895      0.030    723.647      0.000      21.631      21.748
				Q_117923       6.6577      0.035    192.616      0.000       6.590       6.725
				Q_126019      11.1770      0.034    331.536      0.000      11.111      11.243
				Q_15209       15.4162      0.031    501.540      0.000      15.356      15.476
				Q_15454       11.7075      0.034    343.359      0.000      11.641      11.774
				Q_24895       12.9875      0.031    413.256      0.000      12.926      13.049
				Q_27471        6.3829      0.030    210.799      0.000       6.324       6.442
				Q_30420        9.3008      0.033    285.676      0.000       9.237       9.365
				Q_30751       17.9715      0.030    603.191      0.000      17.913      18.030
				Q_30781       20.0875      0.030    679.904      0.000      20.030      20.145
				Q_30836       16.6620      0.031    538.522      0.000      16.601      16.723
				Q_37576       15.2257      0.030    513.126      0.000      15.168      15.284
				Q_39923       17.3462      0.038    460.508      0.000      17.272      17.420
				Q_59786        4.2191      0.034    123.573      0.000       4.152       4.286
				Q_59925        3.8115      0.029    133.686      0.000       3.756       3.867
				Q_66618        7.1930      0.033    217.297      0.000       7.128       7.258
				Q_67554       23.9873      0.030    807.730      0.000      23.929      24.046
				Q_67990        6.8781      0.031    219.870      0.000       6.817       6.939
				Q_81869        8.7419      0.033    267.797      0.000       8.678       8.806
				Q_82880        7.0816      0.031    229.864      0.000       7.021       7.142
				Q_83741       16.6998      0.029    578.695      0.000      16.643      16.756
				Q_84258       31.5590      0.031   1007.481      0.000      31.498      31.620
				Q_85219        8.4527      0.030    279.341      0.000       8.393       8.512
				Q_86433       37.2710      0.033   1135.362      0.000      37.207      37.335
				Q_86767       15.7305      0.031    501.159      0.000      15.669      15.792
				Q_88357        2.3361      0.036     65.187      0.000       2.266       2.406
				Q_88461       11.4151      0.038    303.603      0.000      11.341      11.489
				Q_88518       21.5323      0.031    698.227      0.000      21.472      21.593
				Q_89238        3.2530      0.030    107.585      0.000       3.194       3.312
				Q_95820       37.8954      0.030   1261.341      0.000      37.837      37.954
				Q_96222        3.9435      0.034    114.926      0.000       3.876       4.011
				Q_96226       12.8109      0.033    391.496      0.000      12.747      12.875
				Q_97598       29.6158      0.031    962.402      0.000      29.556      29.676
				==============================================================================
				Omnibus:                   655529.554   Durbin-Watson:                   2.000
				Prob(Omnibus):                  0.000   Jarque-Bera (JB):         30045210.953
				Skew:                          -0.894   Prob(JB):                         0.00
				Kurtosis:                      22.202   Cond. No.                     2.28e+15
				==============================================================================
				
				Notes:
				[1] Standard Errors assume that the covariance matrix of the errors is correctly specified.
				[2] The smallest eigenvalue is 2.33e-24. This might indicate that there are
				strong multicollinearity problems or that the design matrix is singular.
			\end{verbatim} 

			O modelo obteve um $r^2$ de $97,1\%$, ou seja, 97,1\% do modelo pode ser explicado exclusivamente pelo gabarito de respostas do aluno.\footnote{O R-quadrado ou $r^2$ é uma medida estatística de quão próximos os dados estão da linha de regressão ajustada. Ele também é conhecido como o coeficiente de determinação ou o coeficiente de determinação múltipla para a regressão múltipla.}  
			
	\section{Considerações Finais}
	
		\subsection{Tabulação dos resultados}
		
		A partir dos processamentos, ano a ano, dos microdados do ENEM, foi possível tabular as informações na seguinte tabela posição da questão no caderno X Ano de aplicação. Quanto mais vermelho a célula, maior foi o peso dessa questão na nota final do aluno.
		
		% Please add the following required packages to your document preamble:
		% \usepackage[table,xcdraw]{xcolor}
		% If you use beamer only pass "xcolor=table" option, i.e. \documentclass[xcolor=table]{beamer}
		\begin{table}[H]
			\centering
			\begin{tabular}{
					>{\columncolor[HTML]{BDD7EE}}c rrrrrrr}
				{\color[HTML]{2F75B5} \textbf{Peso}}    & \cellcolor[HTML]{DDEBF7}{\color[HTML]{2F75B5} \textbf{Ano}}  & \cellcolor[HTML]{DDEBF7}{\color[HTML]{2F75B5} \textbf{}}     & \cellcolor[HTML]{DDEBF7}{\color[HTML]{2F75B5} \textbf{}}     & \cellcolor[HTML]{DDEBF7}{\color[HTML]{2F75B5} \textbf{}}     & \cellcolor[HTML]{DDEBF7}{\color[HTML]{2F75B5} \textbf{}}     & \cellcolor[HTML]{DDEBF7}{\color[HTML]{2F75B5} \textbf{}}     & \cellcolor[HTML]{DDEBF7}{\color[HTML]{2F75B5} \textbf{}}     \\
				{\color[HTML]{2F75B5} \textbf{Posição}} & \cellcolor[HTML]{DDEBF7}{\color[HTML]{2F75B5} \textbf{2015}} & \cellcolor[HTML]{DDEBF7}{\color[HTML]{2F75B5} \textbf{2016}} & \cellcolor[HTML]{DDEBF7}{\color[HTML]{2F75B5} \textbf{2017}} & \cellcolor[HTML]{DDEBF7}{\color[HTML]{2F75B5} \textbf{2018}} & \cellcolor[HTML]{DDEBF7}{\color[HTML]{2F75B5} \textbf{2019}} & \cellcolor[HTML]{DDEBF7}{\color[HTML]{2F75B5} \textbf{2020}} & \cellcolor[HTML]{DDEBF7}{\color[HTML]{2F75B5} \textbf{2021}} \\
				{\color[HTML]{2F75B5} \textbf{1º}}      & \cellcolor[HTML]{FCDFE2}13,166                               & \cellcolor[HTML]{FCF6F9}3,043                                & \cellcolor[HTML]{FCEAED}8,294                                & \cellcolor[HTML]{FCEAED}8,447                                & \cellcolor[HTML]{FBCCCF}21,892                               & \cellcolor[HTML]{FCE9EB}8,992                                & \cellcolor[HTML]{FCDBDD}15,416                               \\
				{\color[HTML]{2F75B5} \textbf{2º}}      & \cellcolor[HTML]{FBBBBE}29,700                               & \cellcolor[HTML]{FCEFF2}6,008                                & \cellcolor[HTML]{FCEFF2}6,044                                & \cellcolor[HTML]{FCECEF}7,386                                & \cellcolor[HTML]{FCECEF}7,544                                & \cellcolor[HTML]{FA9A9D}44,785                               & \cellcolor[HTML]{FCE4E7}11,177                               \\
				{\color[HTML]{2F75B5} \textbf{3º}}      & \cellcolor[HTML]{FBB4B6}33,073                               & \cellcolor[HTML]{FAA4A6}40,325                               & \cellcolor[HTML]{FBD3D6}18,824                               & \cellcolor[HTML]{FCF5F8}3,226                                & \cellcolor[HTML]{FBD0D2}20,429                               & \cellcolor[HTML]{FCF9FC}1,319                                & \cellcolor[HTML]{FCFCFF}0,000                                \\
				{\color[HTML]{2F75B5} \textbf{4º}}      & \cellcolor[HTML]{FBCCCF}21,835                               & \cellcolor[HTML]{FCDEE1}13,628                               & \cellcolor[HTML]{FCD9DC}15,949                               & \cellcolor[HTML]{FCF4F7}3,922                                & \cellcolor[HTML]{FCD8DB}16,579                               & \cellcolor[HTML]{FA989A}45,867                               & \cellcolor[HTML]{FAA9AC}37,895                               \\
				{\color[HTML]{2F75B5} \textbf{5º}}      & \cellcolor[HTML]{FAAEB0}35,936                               & \cellcolor[HTML]{FCE1E4}12,353                               & \cellcolor[HTML]{FCF8FB}2,158                                & \cellcolor[HTML]{FBD5D7}18,144                               & \cellcolor[HTML]{FA9396}47,879                               & \cellcolor[HTML]{FCEAED}8,432                                & \cellcolor[HTML]{FCE0E3}12,811                               \\
				{\color[HTML]{2F75B5} \textbf{6º}}      & \cellcolor[HTML]{FCF0F3}5,463                                & \cellcolor[HTML]{FCF1F4}5,281                                & \cellcolor[HTML]{FCDFE2}13,472                               & \cellcolor[HTML]{FCF9FC}1,686                                & \cellcolor[HTML]{FCF6F9}2,832                                & \cellcolor[HTML]{FCF4F7}3,614                                & \cellcolor[HTML]{FCD9DC}15,984                               \\
				{\color[HTML]{2F75B5} \textbf{7º}}      & \cellcolor[HTML]{FBC1C4}26,874                               & \cellcolor[HTML]{FBC2C5}26,624                               & \cellcolor[HTML]{FCEAED}8,341                                & \cellcolor[HTML]{FBD4D7}18,195                               & \cellcolor[HTML]{FCEDF0}6,810                                & \cellcolor[HTML]{FBC6C9}24,796                               & \cellcolor[HTML]{FCFCFF}0,308                                \\
				{\color[HTML]{2F75B5} \textbf{8º}}      & \cellcolor[HTML]{FBCCCE}22,175                               & \cellcolor[HTML]{FCF2F5}4,822                                & \cellcolor[HTML]{FCFCFF}0,027                                & \cellcolor[HTML]{FCEFF2}6,227                                & \cellcolor[HTML]{FA9294}48,616                               & \cellcolor[HTML]{FCEAED}8,510                                & \cellcolor[HTML]{FAABAD}37,271                               \\
				{\color[HTML]{2F75B5} \textbf{9º}}      & \cellcolor[HTML]{FCF2F5}4,569                                & \cellcolor[HTML]{FAA2A5}41,029                               & \cellcolor[HTML]{FCE0E3}12,907                               & \cellcolor[HTML]{FA9194}48,846                               & \cellcolor[HTML]{FBBCBE}29,592                               & \cellcolor[HTML]{FCE4E7}10,917                               & \cellcolor[HTML]{FBD6D9}17,346                               \\
				{\color[HTML]{2F75B5} \textbf{10º}}     & \cellcolor[HTML]{FBC1C4}26,993                               & \cellcolor[HTML]{FCE2E5}12,146                               & \cellcolor[HTML]{FCE8EA}9,454                                & \cellcolor[HTML]{FCDBDE}15,056                               & \cellcolor[HTML]{FCFCFF}-0,090                               & \cellcolor[HTML]{FCF5F8}3,399                                & \cellcolor[HTML]{FCF4F7}3,811                                \\
				{\color[HTML]{2F75B5} \textbf{11º}}     & \cellcolor[HTML]{FCDBDD}15,373                               & \cellcolor[HTML]{FBD6D9}17,491                               & \cellcolor[HTML]{FBCDCF}21,709                               & \cellcolor[HTML]{FBC7C9}24,544                               & \cellcolor[HTML]{FCF7FA}2,451                                & \cellcolor[HTML]{FCF8FB}1,789                                & \cellcolor[HTML]{FCF6F9}2,846                                \\
				{\color[HTML]{2F75B5} \textbf{12º}}     & \cellcolor[HTML]{FBCCCF}21,953                               & \cellcolor[HTML]{FCE8EA}9,482                                & \cellcolor[HTML]{FCE6E9}10,236                               & \cellcolor[HTML]{FBB5B7}32,774                               & \cellcolor[HTML]{FCF3F6}4,432                                & \cellcolor[HTML]{FCF6F9}2,826                                & \cellcolor[HTML]{FBB7BA}31,559                               \\
				{\color[HTML]{2F75B5} \textbf{13º}}     & \cellcolor[HTML]{FBD3D6}18,960                               & \cellcolor[HTML]{FBD4D6}18,632                               & \cellcolor[HTML]{FCE0E2}13,127                               & \cellcolor[HTML]{FCE8EB}9,130                                & \cellcolor[HTML]{FCE9EC}8,734                                & \cellcolor[HTML]{FBC8CB}23,929                               & \cellcolor[HTML]{FBD5D8}17,972                               \\
				{\color[HTML]{2F75B5} \textbf{14º}}     & \cellcolor[HTML]{FCE4E7}10,983                               & \cellcolor[HTML]{FCE3E6}11,445                               & \cellcolor[HTML]{FAA8AB}38,317                               & \cellcolor[HTML]{FCE5E8}10,705                               & \cellcolor[HTML]{FAA5A7}39,853                               & \cellcolor[HTML]{FCF5F8}3,364                                & \cellcolor[HTML]{FBBBBE}29,616                               \\
				{\color[HTML]{2F75B5} \textbf{15º}}     & \cellcolor[HTML]{FCDFE2}13,299                               & \cellcolor[HTML]{FCE2E5}11,892                               & \cellcolor[HTML]{FAA1A4}41,592                               & \cellcolor[HTML]{FCE3E6}11,554                               & \cellcolor[HTML]{FBD4D7}18,268                               & \cellcolor[HTML]{FCE8EA}9,473                                & \cellcolor[HTML]{FCE3E5}11,707                               \\
				{\color[HTML]{2F75B5} \textbf{16º}}     & \cellcolor[HTML]{FCDEE1}13,875                               & \cellcolor[HTML]{FCE4E7}10,926                               & \cellcolor[HTML]{FCEAED}8,344                                & \cellcolor[HTML]{FCFBFE}0,413                                & \cellcolor[HTML]{FCD8DB}16,610                               & \cellcolor[HTML]{FCEDF0}7,098                                & \cellcolor[HTML]{FCEAED}8,453                                \\
				{\color[HTML]{2F75B5} \textbf{17º}}     & \cellcolor[HTML]{FCEBEE}7,677                                & \cellcolor[HTML]{FCFAFD}1,195                                & \cellcolor[HTML]{FCEDF0}6,827                                & \cellcolor[HTML]{FCFCFF}-0,100                               & \cellcolor[HTML]{FCF8FB}2,004                                & \cellcolor[HTML]{FCEEF1}6,583                                & \cellcolor[HTML]{FCE9EC}8,742                                \\
				{\color[HTML]{2F75B5} \textbf{18º}}     & \cellcolor[HTML]{FBCFD2}20,512                               & \cellcolor[HTML]{FAB1B3}34,613                               & \cellcolor[HTML]{FCE8EB}9,414                                & \cellcolor[HTML]{FCE5E8}10,632                               & \cellcolor[HTML]{FBCACD}23,011                               & \cellcolor[HTML]{FBBABD}30,198                               & \cellcolor[HTML]{FCE0E3}12,825                               \\
				{\color[HTML]{2F75B5} \textbf{19º}}     & \cellcolor[HTML]{FCF0F3}5,625                                & \cellcolor[HTML]{FBB5B7}32,626                               & \cellcolor[HTML]{F8696B}67,069                               & \cellcolor[HTML]{FCEFF2}6,106                                & \cellcolor[HTML]{FCF9FC}1,521                                & \cellcolor[HTML]{FCF1F4}5,173                                & \cellcolor[HTML]{FCF5F8}3,253                                \\
				{\color[HTML]{2F75B5} \textbf{20º}}     & \cellcolor[HTML]{FCEBEE}7,892                                & \cellcolor[HTML]{FAA7A9}39,029                               & \cellcolor[HTML]{FBBDC0}28,743                               & \cellcolor[HTML]{FBCACC}23,192                               & \cellcolor[HTML]{FCEAED}8,138                                & \cellcolor[HTML]{FCDCDE}14,953                               & \cellcolor[HTML]{FCEEF1}6,505                                \\
				{\color[HTML]{2F75B5} \textbf{21º}}     & \cellcolor[HTML]{FBD6D9}17,490                               & \cellcolor[HTML]{FCE5E7}10,796                               & \cellcolor[HTML]{FCF9FC}1,568                                & \cellcolor[HTML]{FCFBFE}0,592                                & \cellcolor[HTML]{FCFCFF}0,218                                & \cellcolor[HTML]{FBCDD0}21,457                               & \cellcolor[HTML]{FCEDF0}6,878                                \\
				{\color[HTML]{2F75B5} \textbf{22º}}     & \cellcolor[HTML]{FBD0D3}20,221                               & \cellcolor[HTML]{FAB1B3}34,393                               & \cellcolor[HTML]{FBB7B9}31,678                               & \cellcolor[HTML]{FCDCDF}14,633                               & \cellcolor[HTML]{FCF7FA}2,300                                & \cellcolor[HTML]{FCFCFF}0,000                                & \cellcolor[HTML]{FCF4F7}3,943                                \\
				{\color[HTML]{2F75B5} \textbf{23º}}     & \cellcolor[HTML]{FCDCDF}14,566                               & \cellcolor[HTML]{FBCCCF}22,051                               & \cellcolor[HTML]{FCFAFD}1,013                                & \cellcolor[HTML]{FCF0F2}5,810                                & \cellcolor[HTML]{FCF8FB}2,159                                & \cellcolor[HTML]{FCF9FC}1,671                                & \cellcolor[HTML]{FCF3F6}4,219                                \\
				{\color[HTML]{2F75B5} \textbf{24º}}     & \cellcolor[HTML]{FBD6D9}17,407                               & \cellcolor[HTML]{FCF7FA}2,620                                & \cellcolor[HTML]{FBD7D9}17,176                               & \cellcolor[HTML]{F97476}62,431                               & \cellcolor[HTML]{FCF8FB}1,870                                & \cellcolor[HTML]{FCEBEE}8,002                                & \cellcolor[HTML]{FCF7FA}2,336                                \\
				{\color[HTML]{2F75B5} \textbf{25º}}     & \cellcolor[HTML]{FCDADD}15,627                               & \cellcolor[HTML]{FCF1F4}5,344                                & \cellcolor[HTML]{FCEBEE}7,894                                & \cellcolor[HTML]{FCE0E2}13,099                               & \cellcolor[HTML]{FBC8CB}23,825                               & \cellcolor[HTML]{FCF2F4}4,897                                & \cellcolor[HTML]{FBCDCF}21,689                               \\
				{\color[HTML]{2F75B5} \textbf{26º}}     & \cellcolor[HTML]{FCDDE0}14,156                               & \cellcolor[HTML]{FCE8EB}9,154                                & \cellcolor[HTML]{FCD9DC}16,200                               & \cellcolor[HTML]{FCF3F6}4,255                                & \cellcolor[HTML]{FCE9EC}8,770                                & \cellcolor[HTML]{FCF4F7}3,719                                & \cellcolor[HTML]{FCFAFD}1,221                                \\
				{\color[HTML]{2F75B5} \textbf{27º}}     & \cellcolor[HTML]{FCE4E7}11,040                               & \cellcolor[HTML]{FCE4E7}10,872                               & \cellcolor[HTML]{FCFAFD}1,011                                & \cellcolor[HTML]{FBD0D2}20,378                               & \cellcolor[HTML]{FCF3F6}4,192                                & \cellcolor[HTML]{FCE9EC}8,913                                & \cellcolor[HTML]{FBCDD0}21,532                               \\
				{\color[HTML]{2F75B5} \textbf{28º}}     & \cellcolor[HTML]{FCDEE1}13,662                               & \cellcolor[HTML]{FCE7EA}9,798                                & \cellcolor[HTML]{FCDEE1}13,901                               & \cellcolor[HTML]{FCFCFF}0,000                                & \cellcolor[HTML]{FCFBFE}0,627                                & \cellcolor[HTML]{FCE5E7}10,836                               & \cellcolor[HTML]{FCF5F8}3,462                                \\
				{\color[HTML]{2F75B5} \textbf{29º}}     & \cellcolor[HTML]{FCDEE1}13,670                               & \cellcolor[HTML]{FCE2E5}11,958                               & \cellcolor[HTML]{FAAFB1}35,402                               & \cellcolor[HTML]{FBC8CB}23,834                               & \cellcolor[HTML]{FCF5F8}3,321                                & \cellcolor[HTML]{FAA5A8}39,695                               & \cellcolor[HTML]{FBC0C3}27,509                               \\
				{\color[HTML]{2F75B5} \textbf{30º}}     & \cellcolor[HTML]{FBBBBD}29,947                               & \cellcolor[HTML]{FCE5E8}10,529                               & \cellcolor[HTML]{FBD6D9}17,347                               & \cellcolor[HTML]{FCF7FA}2,537                                & \cellcolor[HTML]{FBD5D8}17,788                               & \cellcolor[HTML]{FCF4F7}3,770                                & \cellcolor[HTML]{FCEFF2}6,113                                \\
				{\color[HTML]{2F75B5} \textbf{31º}}     & \cellcolor[HTML]{FCD8DB}16,657                               & \cellcolor[HTML]{FCF4F7}3,738                                & \cellcolor[HTML]{FCF8FB}2,016                                & \cellcolor[HTML]{FCE8EB}9,239                                & \cellcolor[HTML]{FCE0E3}12,696                               & \cellcolor[HTML]{FCF1F4}5,069                                & \cellcolor[HTML]{FCEDEF}7,193                                \\
				{\color[HTML]{2F75B5} \textbf{32º}}     & \cellcolor[HTML]{FBC7C9}24,517                               & \cellcolor[HTML]{FCF7FA}2,256                                & \cellcolor[HTML]{FCF7FA}2,545                                & \cellcolor[HTML]{FCE8EB}9,124                                & \cellcolor[HTML]{FCF6F9}2,856                                & \cellcolor[HTML]{FBCFD2}20,676                               & \cellcolor[HTML]{FCEDF0}7,082                                \\
				{\color[HTML]{2F75B5} \textbf{33º}}     & \cellcolor[HTML]{FCE4E7}10,975                               & \cellcolor[HTML]{FCF4F7}3,887                                & \cellcolor[HTML]{FCE4E7}11,167                               & \cellcolor[HTML]{FBC2C5}26,619                               & \cellcolor[HTML]{FCFAFD}1,190                                & \cellcolor[HTML]{FCE9EC}8,785                                & \cellcolor[HTML]{FCE1E4}12,387                               \\
				{\color[HTML]{2F75B5} \textbf{34º}}     & \cellcolor[HTML]{FBD3D5}19,089                               & \cellcolor[HTML]{FCF5F8}3,273                                & \cellcolor[HTML]{FCE5E8}10,611                               & \cellcolor[HTML]{FCF5F8}3,219                                & \cellcolor[HTML]{FBD5D8}18,018                               & \cellcolor[HTML]{FCEAED}8,466                                & \cellcolor[HTML]{FCE8EB}9,301                                \\
				{\color[HTML]{2F75B5} \textbf{35º}}     & \cellcolor[HTML]{FBD8DA}16,712                               & \cellcolor[HTML]{FCDFE2}13,321                               & \cellcolor[HTML]{FBCACD}23,008                               & \cellcolor[HTML]{FCE8EB}9,112                                & \cellcolor[HTML]{FCEBEE}8,058                                & \cellcolor[HTML]{FCE4E7}10,960                               & \cellcolor[HTML]{FCEEF1}6,383                                \\
				{\color[HTML]{2F75B5} \textbf{36º}}     & \cellcolor[HTML]{FCE1E4}12,266                               & \cellcolor[HTML]{FCF4F7}3,686                                & \cellcolor[HTML]{FCECEF}7,254                                & \cellcolor[HTML]{FBBEC1}28,308                               & \cellcolor[HTML]{FBD1D4}19,655                               & \cellcolor[HTML]{FCE4E7}11,005                               & \cellcolor[HTML]{FCE3E6}11,415                               \\
				{\color[HTML]{2F75B5} \textbf{37º}}     & \cellcolor[HTML]{FBD7DA}17,012                               & \cellcolor[HTML]{FCEFF2}6,037                                & \cellcolor[HTML]{FBD4D7}18,437                               & \cellcolor[HTML]{FAA7A9}39,189                               & \cellcolor[HTML]{FCF4F7}3,720                                & \cellcolor[HTML]{FCEAED}8,294                                & \cellcolor[HTML]{FCDADD}15,730                               \\
				{\color[HTML]{2F75B5} \textbf{38º}}     & \cellcolor[HTML]{FBD1D4}19,711                               & \cellcolor[HTML]{FCDFE1}13,588                               & \cellcolor[HTML]{FCF3F6}4,274                                & \cellcolor[HTML]{FCF7FA}2,194                                & \cellcolor[HTML]{FCF9FC}1,515                                & \cellcolor[HTML]{FCF9FC}1,458                                & \cellcolor[HTML]{FBD0D3}20,087                               \\
				{\color[HTML]{2F75B5} \textbf{39º}}     & \cellcolor[HTML]{FCDADD}15,453                               & \cellcolor[HTML]{FCECEE}7,630                                & \cellcolor[HTML]{FBD6D8}17,664                               & \cellcolor[HTML]{FCEFF2}6,177                                & \cellcolor[HTML]{FCE7EA}9,610                                & \cellcolor[HTML]{FCE2E5}12,058                               & \cellcolor[HTML]{FBD8DA}16,700                               \\
				{\color[HTML]{2F75B5} \textbf{40º}}     & \cellcolor[HTML]{FCF2F5}4,747                                & \cellcolor[HTML]{FCFCFF}0,314                                & \cellcolor[HTML]{FCF3F6}4,142                                & \cellcolor[HTML]{FCFAFD}1,144                                & \cellcolor[HTML]{FCFBFE}0,769                                & \cellcolor[HTML]{FBD3D6}18,925                               & \cellcolor[HTML]{FCD9DC}16,127                               \\
				{\color[HTML]{2F75B5} \textbf{41º}}     & \cellcolor[HTML]{FAA6A9}39,360                               & \cellcolor[HTML]{FCE5E8}10,717                               & \cellcolor[HTML]{FCF4F7}3,675                                & \cellcolor[HTML]{FCD9DC}15,906                               & \cellcolor[HTML]{FCF6F9}2,750                                & \cellcolor[HTML]{FBCACD}22,872                               & \cellcolor[HTML]{FBC8CA}23,987                               \\
				{\color[HTML]{2F75B5} \textbf{42º}}     & \cellcolor[HTML]{FBB8BB}31,062                               & \cellcolor[HTML]{FBB4B6}33,233                               & \cellcolor[HTML]{FCDCDE}14,913                               & \cellcolor[HTML]{FBD4D6}18,579                               & \cellcolor[HTML]{FCF5F8}3,517                                & \cellcolor[HTML]{FAA9AB}38,126                               & \cellcolor[HTML]{FCDBDE}15,226                               \\
				{\color[HTML]{2F75B5} \textbf{43º}}     & \cellcolor[HTML]{FBCCCE}22,140                               & \cellcolor[HTML]{FCE1E4}12,464                               & \cellcolor[HTML]{FAB1B3}34,467                               & \cellcolor[HTML]{FCF6F9}2,799                                & \cellcolor[HTML]{FCE8EB}9,324                                & \cellcolor[HTML]{FBCDD0}21,642                               & \cellcolor[HTML]{FCD8DB}16,662                               \\
				{\color[HTML]{2F75B5} \textbf{44º}}     & \cellcolor[HTML]{FCE3E6}11,437                               & \cellcolor[HTML]{FAACAE}36,812                               & \cellcolor[HTML]{FCF1F4}5,329                                & \cellcolor[HTML]{FCD9DC}16,134                               & \cellcolor[HTML]{FBC1C4}27,071                               & \cellcolor[HTML]{FCE1E4}12,503                               & \cellcolor[HTML]{FCEEF1}6,658                                \\
				{\color[HTML]{2F75B5} \textbf{45º}}     & \cellcolor[HTML]{FAAFB1}35,467                               & \cellcolor[HTML]{FCF6F9}3,056                                & \cellcolor[HTML]{FCE3E5}11,768                               & \cellcolor[HTML]{FCDEE1}13,717                               & \cellcolor[HTML]{FAAEB0}35,780                               & \cellcolor[HTML]{FBD7DA}16,972                               & \cellcolor[HTML]{FCE0E3}12,987                              
			\end{tabular}
		\end{table}
		
		Para melhor compreensão, a 19º questão de 2017 registrou um peso de 67,069 pontos, a maior da série histórica. Isso quer dizer que o aluno que eventualmente acertou essa questão, ganhou 67 pontos na nota final. Nesse ano a média da matéria Matemática e suas Tecnologias registrou uma média de 518,5, o que significa que com essa única questão, entre as 45, permitiu o aluno ganhar 13\% da nota da prova.
		
		Isso está de acordo com as diretrizes da TRI, em que o aluno com habilidade $\theta$ tende a ter muito mais chance de acertar questões que necessitam de pouca habilidade.
		
		Como curiosidade, a 19º questão de 2017 mapeada no banco de dados é essa da figura~\ref{fig:0004-questao}
		
		\begin{figure}[H]
			\centering
			\includegraphics[width=0.99\linewidth]{figuras/0004-Questão.Peso.67}
			\caption{}
			\label{fig:0004-questao}
		\end{figure}
		
		É possível notar que 59,2\% dos alunos acertaram a questão.
					

	\section{Conclusões}
	
		Esses dados permitem mapear as questões mais fáceis e mostrar aos alunos que o treino e a disciplina de estudos podem facilitar o entendimento, as vezes não trivial, que acertar as questões consideradas fáceis permitem uma melhor nota.
		
		Embora o TRI tenha um esquema não linear, é possível a partir dos dados aplicar a TCT nas provas do ENEM e obter o eventual score que ele teria caso fizesse aquela prova.
		
		Está feito o convite para ao leitor se debruçar sobre os dados e debater eventuais novas conclusões.
		
		
	% ----------------------------------------------------------
	% ELEMENTOS PÓS-TEXTUAIS
	% ----------------------------------------------------------
	\postextual
	
	% ----------------------------------------------------------
	% Referências bibliográficas
	% ----------------------------------------------------------
	\bibliography{../02-defesa/bibliografia/!!!bibliografia}
	
	% ----------------------------------------------------------
	% Glossário
	% ----------------------------------------------------------
	%
	% Há diversas soluções prontas para glossário em LaTeX. 
	% Consulte o manual do abnTeX2 para obter sugestões.
	%
	%\glossary
	
	% ----------------------------------------------------------
	% Apêndices
	% ----------------------------------------------------------
	
	% ---
	% Inicia os apêndices
	% ---
	\ImprimirSimOuNao[Não]{
		\begin{apendicesenv}
			
			% ----------------------------------------------------------
			\chapter{Nullam elementum urna vel imperdiet sodales elit ipsum pharetra ligula
				ac pretium ante justo a nulla curabitur tristique arcu eu metus}
			% ----------------------------------------------------------
			\lipsum[55-56]
			
		\end{apendicesenv}
	}
	% ---
	
	% ----------------------------------------------------------
	% Anexos
	% ----------------------------------------------------------
	\ImprimirSimOuNao[Não]{
		\cftinserthook{toc}{AAA}
		% ---
		% Inicia os anexos
		% ---
		%\anexos
		\begin{anexosenv}
			
			% ---
			\chapter{Cras non urna sed feugiat cum sociis natoque penatibus et magnis dis
				parturient montes nascetur ridiculus mus}
			% ---
			
			\lipsum[31]
			
		\end{anexosenv}
	}
	
	% ----------------------------------------------------------
	% Agradecimentos
	% ----------------------------------------------------------
	
	\ImprimirSimOuNao[Não]{
		\section*{Agradecimentos}
		Texto sucinto aprovado pelo periódico em que será publicado. Último 
		elemento pós-textual.
	}
	
\end{document}